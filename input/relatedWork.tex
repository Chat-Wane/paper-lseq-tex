\section{Related Work}
\label{sec:relatedwork}

Popular distributed collaborative editors such as Google
Docs~\cite{nichols1995high} rely on Operational Transformation approach
(OT) \cite{sun1998operational,sun1998achieving}.  OT-based and CRDT-based
distributed editors follow the same global scheme of optimistic replication,
i.e., generate operations without locking, broadcast to others replicates and
re-execute. OT and CRDT mainly differ in their complexities:
\begin{inparaenum}[(i)]
\item OT-based editors have constant-time complexity at generation time and a
  complexity of $O(|H^{2}|)$ at re-execution time where $H$ is the log of
  operations. Performance of OT closely depend on the number of concurrent
  operations present in the system.
\item LSEQ sequence CRDT has a complexity of $O(k)$ for generation time and
  $O(k*log(n))$ for re-execution time where $n$ is the number of elements
  presents in the document and $k$ is proportional to size of
  identifiers. Unlike OT, the state of the document mainly determines the CRDT
  performance. \NAME{} significantly improves the performance of the sequence
  CRDTs by keeping $k$ small.
\end{inparaenum}

The tombstone class of sequence CRDT includes WOOT \cite{oster2006data},
WOOTO~\cite{weiss2007wooki}, WOOTH~\cite{ahmed2011evaluating},
Treedoc~\cite{preguica2009commutative}, CT~\cite{grishchenko2010deep},
RGA \cite{roh2011replicated}, \cite{Yu2012stringwise}. In these approach,
tombstones (or ``death certificates'') mark the deleted elements. They provide
a simple solution to solve problems of concurrent delete. A clear advantage is
to only require fixed-length identifiers, nonetheless the space complexity of
tombstone-based sequence CRDTs is linear compared to the number of insert
operations performed on the document.

Safely garbaging tombstones in a distributed system is costly because it
requires obtaining a consensus for this decision among all participants.
In~\cite{roh2011replicated, letia2009crdts}, they proposed some solutions
related to the garbage collecting mechanism in order to rebalance and/or purge
the model of the CRDT.  The purge~\cite{roh2011replicated} of tombstones
requires a full vector clock to keep track of updates on other replicas and to
be able to safely remove the tombstones. The \emph{core
  nebula}~\cite{letia2009crdts} approach intends to make the consensus
reachable, but constrains the topology of the network and uses an expensive
\emph{catch up} algorithm.

The variable-size identifiers class of CRDT includes
Logoot~\cite{weiss2009logoot} and Treedoc~\cite{preguica2009commutative}. These
CRDTs use growing identifiers to encode the total order among elements of the
sequence. In the worst case, the size of identifiers is linear in the total
number of insert operations done on the
document~\cite{ahmed2011evaluating}. Logoot and
Treedoc~\cite{preguica2009commutative} have different allocation
strategies. Treedoc has two allocation strategies:
\begin{inparaenum}[(i)]
\item the first strategy allocates an identifier by directly appending a bit on
  one of its neighbour identifier.
\item The second strategy increases the depth of this new identifier by $\lceil
  log_2(h)\rceil +1 $ (where $h$ is the highest depth of the identifiers
  already allocated) and allocates the lowest value possible with this growth,
  in prevision of future insertions.
\end{inparaenum}

Logoot's \emph{boundary} strategy and Treedoc's second strategy are very
similar, both in their goals and their weaknesses. They assume an editing
behaviour in the end, and therefore they become application dependent.
Compared to Logoot and Treedoc, \NAME{} is adaptive and significantly enlarges
the applicability of sequence CRDTs.

In~\cite{ahmed2011evaluating}, they compared most sequence CRDTs and one OT in
an experimental setup. RGA and Logoot obtained best overall performances. In
this paper, we completed experiments with more extreme cases and demonstrated
that \NAME{} outperforms Logoot.

