

\begin{abstract}
  Distributed collaborative editing systems allow users to work distributed in
  time, space and across organizations. Trending distributed collaborative
  editors such as Google Docs, Etherpad or Git have grown in popularity over
  the years. A new kind of distributed editors based on a family of distributed
  data structure replicated on several sites called Conflict-free Replicated
  Data Type (CRDT for short) appeared recently. This paper considers a CRDT
  that represents a distributed sequence of basic elements that can be lines,
  words or characters (sequence CRDT). The possible operations on this sequence
  are the insertion and the deletion of elements. Compared to the state of the
  art, this approach is more decentralized and better scales in terms of the
  number of participants. However, its space complexity is linear with respect
  to the total number of inserts and the insertion points in the document. This
  makes the overall performance of such editors dependent on the editing
  behaviour of users. This paper proposes and models \NAME{}, an adaptive
  allocation strategy for a sequence CRDT. \NAME{} achieves in the average a
  sub-linear spatial-complexity whatever is the editing behaviour. A series of
  experiments validates \NAME{} showing that it outperforms existing
  approaches.
\end{abstract}
