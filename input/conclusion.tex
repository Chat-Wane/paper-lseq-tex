\section{Conclusion}

In this paper, we presented an original allocation strategy for sequence CRDTs
called \NAME{}. Compared to state of art, \NAME{} is adaptive, i.e., it handles
unpredictable different editing behaviour and achieves sub-linear space
complexity. Consequently \NAME{} does not require a costly protocol to garbage
or re-balance identifiers, and is suitable for building better distributed
collaborative editors based on sequence CRDTs.

Three components compose \NAME{}:
\begin{inparaenum}[(1)]
  \item a base doubling,
  \item two allocation strategies \emph{boundary+} and \emph{boundary--},
  \item a random strategy choice.
\end{inparaenum}

Although each component cannot achieve sub-linear complexity, the conjunction
of three components provides the expected behaviour. Experiments show that
even if \NAME{} makes a bad strategy choice for one level in the tree, this
choice will be overwhelmed by the gain obtained at next levels.


The \NAME{} approach is generic enough to be included in other variable-size
sequence CRDTs. Current experiments were done with a Logoot basis because it
does not require tombstones and therefore is less dependent of the editing
behaviour. But we believe that Treedoc's heuristic could be improved with this
allocation strategy.

Future works include a formal demonstration of the empiric poly-logarithmic
upper-bound in space complexity which implies a probabilistic study of the
worst-case. The idea is to prove that its probability of happening is
negligible. We also plan to study if concurrency affects LSEQ results, i.e., if
each site makes different allocation choices concurrently, does it impact
\NAME{} performances? Finally, we aim to study if using documents spectrum
knowledge and machine-learning approaches can outperform random strategy
choice.
