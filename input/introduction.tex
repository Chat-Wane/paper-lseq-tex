\section{Introduction}

Distributed collaborative editing
systems~\cite{Ellis:1989:CCG:67544.66963,ellis1991groupware,greenberg1994real}
such as Google Docs, Etherpad or Git are now widely used and allow users to
work distributed in time, space, and across organizations. A new kind of
distributed editors~\cite{preguica2009commutative,weiss2007wooki} appeared
based on Conflict-Free Replicated Data Types
(CRDTs)~\cite{oster2006data,weiss2009logoot,shapiro2011conflict}. A CRDT is a
distributed data type replicated over several
sites~\cite{saito2005optimistic,saito2002replication}. A CRDT cannot implement
any centralized data structure but for instance can implement a counter, a set,
a tree, etc. In this paper we consider a special family or CRDTs that implement
a sequence of basic elements such as lines, words or characters that we call
\emph{sequence CRDT}. For our purpose and as a first step, we only consider two
basic operations on a sequence, the insert and the delete operations. Compared
to the state of the art, editors based on sequence CRDTs are more decentralized
and scale better. However, they have a linear space-complexity with respect to
the number of insertions. Consequently, they heavily depend on the editing
behaviour. Some editing scenarios lead to a permanent loss in performance.

In order to preserve the total order on the elements of the sequence, a unique
and immutable identifier is associated with each basic element of the structure
(character, line or paragraph according to the chosen granularity). This allows
distinguishing two classes of sequence CRDTs:
\begin{inparaenum}[(i)]
\item Fixed size identifier (also called the tombstones class). This class
  includes WOOT~\cite{oster2006data}, WOOTO~\cite{weiss2007wooki},
  WOOTH~\cite{ahmed2011evaluating}, CT~\cite{grishchenko2010deep},
  RGA~\cite{roh2011replicated},~\cite{Yu2012stringwise}. In this class, a
  tombstone replaces each suppressed element. Although it enjoys a fixed length
  for identifiers and it has a space complexity which depends on the number of
  operations. For example, a document with an history of a million operations
  and finally containing a single line can have as much as 499999
  tombstones. Garbaging tombstones requires costly protocols in decentralized
  distributed systems.
\item Variable-size identifiers. This class includes for example
  Logoot~\cite{weiss2009logoot}. It does not require tombstones, but its
  identifiers can grow unbounded. Consequently, although it does not require
  garbage protocols, its space complexity remains till now linear with the
  number of insert operations. Thus, it is possible to have only a single
  element in the sequence having an identifier of length
  499999. Treedoc~\cite{preguica2009commutative} uses both tombstones and
  variable size identifiers but relies on a complex garbage protocol when
  identifiers grow too much.
\end{inparaenum}

In this paper, we propose a new approach, called \NAME{}, that belongs to the
variable-size identifiers class of sequence CRDTs. Compared to the state of the
art, \NAME{} is an adaptive allocation strategy with a sub-linear upper-bound
in its spatial complexity. We experimented \NAME{} on synthetic sequences and
real documents. In both cases, \NAME{} outperforms existing approaches.

The remainder of the paper is organized as
follows. Section~\ref{sec:background} delineates the background on
variable-size identifiers class of sequence CRDTs and motivates this
work. Then, Section~\ref{sec:proposal} details \NAME{} and the three parameters
that control the growth and the selection of the identifiers. It describes each
parameter with its aim and its defect. Also, how their composition overcomes
their respective weaknesses. Section~\ref{sec:validation} validates the
approach by showing the effect of each one of the three parameters of \NAME{}
and also by comparing the proposed approach to Logoot. Finally,
Section~\ref{sec:relatedwork} reviews related works.
